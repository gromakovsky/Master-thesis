\documentclass[russian, hyperref={unicode}]{beamer}

\usetheme{Madrid}
\setbeamertemplate{footline}[frame number]
\setbeamertemplate{navigation symbols}{}
\setbeamerfont{note page}{size=\footnotesize}
\setbeamertemplate{note page} {%
  \hbox{\insertshortframetitle[width=8cm]}%
  \vskip1.75em
  \nointerlineskip
  \insertnote
}

% \setbeameroption{show only notes}

\usepackage[T2A]{fontenc}
\usepackage{lmodern}
\usepackage[utf8x]{inputenc}
\usepackage[english, russian, main=russian]{babel}
\usepackage{graphicx}

\usepackage{algpseudocode}
\algtext*{EndWhile}     % Remove "end while" text
\algtext*{EndFor}       % Remove "end for" text
\algtext*{EndFunction}  % Remove "end function" text
\algtext*{EndIf}        % Remove "end if" text

\graphicspath{ {../Images/} }

% workaround warning
\let\Tiny=\tiny

\title{Разработка алгоритмов статического поиска выходов за пределы
  динамического массива в С/C++ программах}
\author{И. Е. Громаковский \\
  {\small Научный руководитель: М. А. Лукин}}
\institute{Санкт-Петербургский национальный исследовательский
  университет \\ информационных технологий, механики и оптики}
\date{}

\begin{document}

\section{Введение}

\frame{\titlepage}
\note{Здравствуйте!}

\subsection{Решаемая проблема}

\begin{frame}{Описание задачи}
    \begin{itemize}
        \item Что-то.
        \item Что-то ещё.
    \end{itemize}
\end{frame}
\note {
  Патак.
}

\section{Результаты}

\begin{frame}{Результаты}
    \begin{itemize}
        \item Умею
        \item Могу
    \end{itemize}
\end{frame}
\note {
  Типа вот вам результаты, всё круто, я хорош.
}

\begin{frame}{Спасибо за внимание!}
    \begin{center}
        \Huge
        {\color{blue} Вопросы?}
    \end{center}
\end{frame}
\note{До свидания!}

\appendix

\begin{frame}[noframenumbering, t]{Обоснование метрик}
    \only<1> {
      kek
    }
    \only<2> {
      хи-хи
    }
\end{frame}

\end{document}
