\documentclass[russian, hyperref={unicode}]{beamer}

\usetheme{Madrid}
\setbeamertemplate{footline}[frame number]
\setbeamertemplate{navigation symbols}{}
\setbeamerfont{note page}{size=\footnotesize}
\setbeamertemplate{note page} {%
  \hbox{\insertshortframetitle[width=8cm]}%
  \vskip1.75em
  \nointerlineskip
  \insertnote
}

% \setbeameroption{show only notes}

\usepackage[T2A]{fontenc}
\usepackage{lmodern}
\usepackage[utf8x]{inputenc}
\usepackage[english, russian, main=russian]{babel}
\usepackage{graphicx}
\usepackage{tikz}

\usepackage{algpseudocode}
\algtext*{EndWhile}     % Remove "end while" text
\algtext*{EndFor}       % Remove "end for" text
\algtext*{EndFunction}  % Remove "end function" text
\algtext*{EndIf}        % Remove "end if" text

\graphicspath{ {../Images/} }

\newcommand{\inputTikZ}[1]{\input{../Images/#1.tikz}}

% workaround warning
\let\Tiny=\tiny

\title{Разработка алгоритмов статического поиска выходов за пределы
  динамического массива в С/C++ программах}
\author{И. Е. Громаковский \\
  {\small Научный руководитель: М. А. Лукин}}
\institute{Санкт-Петербургский национальный исследовательский
  университет \\ информационных технологий, механики и оптики}
\date{}

\begin{document}

\section{Введение}

\frame{\titlepage}
\note{Здравствуйте!}

\subsection{Решаемая проблема}

\begin{frame}{Описание задачи}
    \begin{itemize}
        \item Статический поиск выходов за пределы динамического
          массива в C и C++.
        \item Работа с большими программами за разумное время.
        \item В общем случае задача поиска всех ошибок неразрешима.
        \item Находить как можно больше ошибок, минимизируя число
          ложных срабатываний.
    \end{itemize}
\end{frame}
\note { Как видно из названия, в рамках данной работы требуется
  находить ошибки, связанные с выходами за пределы динамического
  массива в программах на C и C++. Делать это нужно статически, что
  позволяет найти ошибки до того, как они проявят себя. Требуется
  уметь обрабатывать большие программы (состоящие из сотен тысяч
  строк) за разумное время (порядка десятков минут). В общем случае
  задача поиска всех выходов за пределы массива неразрешима, поэтому
  неформально задача состоит в поиске как можно большего числа ошибок
  с максимальной точностью.  }

\begin{frame}{Актуальность проблемы}
    \begin{itemize}
        \item Программное обеспечение всегда было и остаётся
          подвержено ошибкам в коде.
        \item Выход за пределы массива — одна из главных уязвимостей с
          точки зрения безопасности.
        \item Особенно опасны ошибки в операционных системах и сетевых
          программах, для которых популярны языки C и C++.
    \end{itemize}
\end{frame}
\note { Данная проблема является крайне актуальной, поскольку
  программное обеспечение всегда было и остаётся подвержено ошибкам в
  коде. Известно большое число примеров, когда ошибки в коде приводили
  к ужасным последствиям. Ошибки, ведущие к выходу за пределы массива
  в C/C++, являются одной из главных уязвимостей с точки зрения
  безопасности, поскольку злоумышленник может, используя такие
  уязвимости, исполнить произвольный код и получить полный контроль
  над системой. Стоит отметить, что особую опасность представляют
  такие ошибки в коде операционных систем и сетевых программ.  Они
  зачастую пишутся на C/C++, что усиливает актуальность проблемы.  }

% \subsection{Обзор существующих решений}

% \begin{frame}{Существующие решения}
%     \begin{itemize}
%         \item Использование аннотаций (Dor et al., 2003)
%         \item Вывод ограничений
%           \begin{itemize}
%             \item Обход путей программы, построение логического
%               утверждения (Ganapathy et al., 2003)
%             \item Анализ по требованию (Le et al., 2008)
%           \end{itemize}
%         \item Анализ промежуточного представления (Li et al., 2010)
%     \end{itemize}
% \end{frame}
% \note {
%   \tiny
%   Среди существующих решений можно выделить некоторые общие
%   подходы. Например, некоторые подходы используют пользовательские
%   аннотации для упрощения проверки корректности доступа к
%   памяти. Однако для больших программ такой подход требует слишком
%   больших человеческих усилий для написания аннотаций, поэтому он
%   практически не применим. Большинство известных подходов выводят
%   ограничения различных значений в программе и делают вывод о наличии
%   ошибок, исходя из этих ограничений. Один из вариантов заключается в
%   обходе всех путей программы и построении системы, ограничивающей
%   возможные значения переменных, а также логического утверждения,
%   соответствующего отсутствию ошибок. После этого используется
%   SMT-solver для проверки выполнимости этого утверждения. При большом
%   числе ветвлений такой подход работает неприемлемо. Более новой
%   является идея анализа по требованию, суть которого
%   состоит в том, чтобы рассматривать только пути программы, которые
%   влияют на обращения к массиву. Вместо описания программы в целом,
%   каждое обращение обрабатывается отдельно, что на практике
%   существенно уменьшает объём вычислений. Также стоит отметить
%   использование промежуточного представления, например, представление
%   LLVM, для анализа.
% }

\subsection{Основа работы}

\begin{frame}{Анализ LLVM-IR}
\begin{itemize}
  \item Меньше языковых конструкций, проще анализ.
  \item Анализируемый код ближе к реально выполняемому.
  \item Встроенные оптимизации и средства для анализа кода.
  \item Автоматически поддерживается любой язык, для которого есть
    компилятор в LLVM-IR.
  \item Static Single Assignment.
\end{itemize}
\end{frame}
\note {
  В данной работе было решено производить анализ на уровне
  промежуточного представления LLVM. Это имеет несколько
  преимуществ. Во-первых, в промежуточном представлении меньше
  языковых конструкций, что упрощает анализ. Во-вторых, анализируемый
  код ближе к реально выполняемому, что позволяет учесть влияние
  компилятора. Также в LLVM имеется большое число встроенных
  оптимизаций и средств для анализа кода, которые могут быть
  использованы для решения подзадач. Анализатор LLVM IR позволяет
  анализировать любой язык, для которого есть компилятор в LLVM IR, в
  частности C/C++. Ещё одним преимуществом является то, что программа
  в LLVM IR представлена в static single assignment форме, которая
  крайне удобна для анализа.
}

\begin{frame}{Li, Cifuentes, Keynes: ranges}
  \begin{itemize}
    \item Li L., Cifuentes C., Keynes N. Practical and Effective
      Symbolic Analysis for Buffer Overflow Detection.
    \item Symbolic ranges.
      \begin{itemize}
        \item Symbolic expressions: $15$, $x + 1$, $\bot$, $\top$.
        \item Частичный порядок.
        \item Symbolic range: $[s1, s2]$.
        \item Операции: $\cup, \cap, +, -, \times, \div$.
      \end{itemize}
    \item Define range, $S_v$:
      \begin{itemize}
        \item в месте определения $V$.
      \end{itemize}
    \item Use range, $S_{V, P}$:
      \begin{itemize}
        \item в произвольном месте $P$.
      \end{itemize}
    \item Выход за пределы массива размера $n$ в инструкции $P$:
      \begin{itemize}
        \item $S_{n, P}^{max} \prec S_{index, P}^{max} \vee S_{index,
            P}^{min} \prec -1$.
      \end{itemize}
  \end{itemize}
\end{frame}
\note {
  В основе данной работы лежит алгоритм, описанный в статье, название
  которой представлено на слайде. В их работе вводится понятие symbolic
  range. Сначала определяется symbolic expression, которым может быть
  константа, переменная, аффиная функция от других symbolic
  expressions, а также bot and top, соответствующие минимальному и
  максимальному значению. На symbolic expressions естественным образом
  задаётся частичный порядок. Symbolic range — пара из symbolic
  expressions. Для symbolic range вводятся операции объединения,
  пересечения, сложения, вычитания, умножения, деления.  Для каждой
  переменной вводится понятие define range — диапазон её значений в
  месте определения.  Также вводится понятие use range — диапазон
  значений переменной в произвольной точке программы. Этот диапазон
  может быть меньше define range, т. к. в некоторые инструкции в
  программе можно прийти только по условным переходам. Например, если
  инструкция под $if (x < 0)$, то в ней $x$ точно меньше нуля.
  Анализатор сообщает о выходе за пределы массива, если максимально
  возможное значение размера может быть меньше либо равно значения
  индекса или если индекс может быть отрицательным.
}

\begin{frame}{Li, Cifuentes, Keynes: зависимости}
  \begin{itemize}
    \item Зависимости по данным.
      \begin{itemize}
        \item Define range вычисляется через диапазоны аргументов.
        \item $S_{(a + b)} = S_{a, P} + S_{b, P}, P = a + b$
        \item $S_{(\phi(a, b))} = S_{a, P} \cup S_{b, P}, P = \phi(a, b)$
        \item …
      \end{itemize}
    \item Зависимости потока управления.
      \begin{itemize}
        \item Уточнение диапазона $S_{V, P}$ на основании условных переходов на пути к $P$.
        \item Условные переходы, связанные с $V$, такие что:
          \begin{itemize}
            \item $P$ строго доминируется условным переходом;
            \item $P$ достижима только из одного потомка условного перехода.
          \end{itemize}
      \end{itemize}
  \end{itemize}
\end{frame}
\note {
  Для расчёта диапазонов значений рассматриваются два вида
  зависимостей.
  Зависимости по данным используются для вычисления define
  range. Например, define range суммы является суммой use ranges
  аргументов в инструкции сложения. Define range фи-инструкции
  является объединением use ranges, т. к. фи может вернуть любое из значений.
  Зависимости потока управления используются для уточнения диапазона
  значений в инструкции $P$ на основании предикатов на пути к $P$ по
  графу потока управления. Для уточнения диапазона значений $V$
  используются предикаты, затрагивающие $V$, такие что выполнены два
  условия. Дальше по слайду.
}

\section{Улучшения}

\subsection{Обработка циклов}

\begin{frame}{Монотонно изменяющиеся переменные}
    \begin{columns}
        \column{.5\textwidth}
            \begin{figure}
                \includegraphics[width=\textwidth, clip=true]{for-trivial.png}
            \end{figure}
        \column{.5\textwidth}
            \begin{figure}
                \inputTikZ{for-trivial}
            \end{figure}
    \end{columns}
    \begin{itemize}
      \item Нет предикатов, ограничивающих значение снизу.
      \item Если $x = \phi(a, b), b = f(x)$ и последовательность $x,
        f(x), f(f(x)) \ldots$ монотонна (например, возрастает), то
        применяется предикат $x \geq a (x \leq a)$.
    \end{itemize}
\end{frame}
\note {
  У описанного в статье подхода есть определённые недостатки,
  которым было уделено внимание в моей работе. Рассмотрим пример на
  слайде. В данном случае нет предиката, ограничивающего значение $x$
  снизу. Алгоритм будет работать следующим образом: поскольку значение
  $x$ зависит от $x_1$, которое в свою очередь зависит от $x$, то
  первоначально в качестве значения $x$ будет взят отрезок
  $[\bot, \top]$. Затем будет учтён предикат, что $x < 10$ в месте
  вычисления $x_1$. В результате диапазон значений $x_1$ будет
  $[\bot, 10]$, диапазон значений $x$ будет такой же. Таким образом
  использование $x$ в качестве индекса будет считаться ошибочным,
  поскольку анализатор посчитает, что он может быть
  отрицательным. Однако нетрудно видеть, что это не так.  Для решения
  этой проблемы вводится правило, описанное на слайде.
}

\begin{frame}{Обработка предиката «не равно»}
    \begin{columns}
        \column{.5\textwidth}
            \begin{figure}
                \includegraphics[width=\textwidth, clip=true]{for-ne.png}
            \end{figure}
        \column{.5\textwidth}
            \begin{figure}
                \inputTikZ{for-ne}
            \end{figure}
    \end{columns}
    \begin{itemize}
      \item Алгоритм из статьи учитывает $x != y$, только если значение на
        границе отрезка.
      \item Если $x = \phi(a, f(x)), \exists c: y = f^c(a)$ и
        последовательность $x, f(x), f(f(x)), \ldots$ монотонна
        (например, возрастает), то применяется предикат $x < y$.
    \end{itemize}
\end{frame}
\note {
  Следующее улучшение связано с обработкой условия «не
  равно». Согласно алгоритму из статьи, условие $x \neq y$ учитывается
  только в том случае, если равенство $x = y$ выполняется для
  граничного значения рассматриваемой переменной. Из-за этого алгоритм
  неспособен корректно обрабатывать циклы, в которых значение счётчика
  ограничено таким предикатом, хотя они встречаются в программах
  крайне часто. Например, в простейшем цикле, представленном на
  рисунке, значение переменной $x$ в месте записи в массив не будет
  ограничено, поскольку число $10$ не лежит на границе отрезка
  значений $x$. Для решения этой проблемы вводится следующее
  правило. Если $x$ является $\phi$ функцией от какого-то значения $a$
  и другого значения, представленного функцией от $x$, и существует
  такая константа $c$, что $c$ последовательных применений $f$ к $a$
  возвращают $y$, а также последовательность применений $f$ к
  аргументу монотонна, то применяется предикат $x < y$ (или $x > y$ в
  зависимости от характера монотонности).
}

% \begin{frame}{Улучшение учёта предикатов}
%     \begin{columns}
%         \column{.5\textwidth}
%             \begin{figure}
%                 \includegraphics[width=\textwidth, clip=true]{predicates-improvement.png}
%             \end{figure}
%         \column{.5\textwidth}
%             \begin{figure}
%                 \inputTikZ{predicates-improvement}
%             \end{figure}
%     \end{columns}
%     \begin{itemize}
%       \item Запись в массив достижима из обоих потомков условного
%         перехода.
%       \item Решение: игнорировать рёбра из условного перехода в
%         других потомков.
%     \end{itemize}
% \end{frame}
% \note {
%   Ещё одно улучшение затрагивает правила для учёта
%   предикатов. Рассмотрим пример на слайде, здесь значение $x$ внутри
%   цикла дополнительно ограничено семью, что делает запись в массив
%   корректной. Однако согласно алгоритму из статьи предикат,
%   ограничивающий значение $x$, должен быть учтён только в том случае,
%   если запись в массив достижима только из одного потомка условного
%   перехода. Нетрудно видеть, что в данном случае запись в массив
%   достижима из обоих потомков, из-за чего предикат не будет учтён, что
%   приведёт к false positive. Для решения проблемы предлагается
%   модификация этого правила, состоящая в игнорировании рёбер из
%   условного перехода в других потомков при проверке наличия путей из
%   потомков.
% }

\begin{frame}{Межпроцедурный анализ}
    \begin{columns}
        \column{.45\textwidth}
            \begin{figure}
                \includegraphics[width=\textwidth, clip=true]{archer-linux.png}
            \end{figure}
        \column{.5\textwidth}
            \begin{itemize}
              \item $32 \prec S^{max}_{di, P}$.
              \item $S^{min}_{di, P} \prec -1$.
              \item $P$ --- инструкция вызова.
            \end{itemize}
    \end{columns}

    \begin{itemize}
      \item Триггер --- условие ошибки: $e_1 \prec e_2$.
      \item Функции анализируются от вызываемой к вызывающей.
      \item Построение триггеров при анализе обращения к массиву.
      \item Проверка триггеров при анализе вызова функции.
    \end{itemize}
\end{frame}
\note {
  Ещё одним недостатком описанного в статье алгоритма является
  отсутствие межпроцедруного анализа. В моей работе была реализована
  поддержка межпроцедурного анализа за счёт использования
  триггеров. Триггер представляет условие, выполнение которого
  соответствует ошибке в программе, и состоит из двух символьных
  выражений. Функции анализируются в порядке от вызываемой к
  вызывающей. При анализе обращения к массиву происходит добавление
  триггера для функции, если корректность обращения зависит от
  аргументов функции. При анализе вызова функции происходит проверка
  триггера. В примере на слайде при анализе get-drv-by-nr будет
  добавлено два триггера, представленных на слайде. При анализе вызова
  этой функции из get-slot-by-minor будет произведена проверка
  триггеров, и один из них сработает, что приведёт к обнаружению
  ошибки. Также вместе с триггером хранится инструкция, в которой
  возможна ошибка.
}

\section{Результаты}

\begin{frame}{Сравнение: Multi Theft Auto}
  Multi Theft Auto 1.3.1: 760 тыс. строк кода, 182 мегабайта биткода.

  \begin{tabular}{|*{18}{c|}}\hline
  --                & TP  & FP  & Время работы (м:с) \\\hline
  Cppcheck          & 1   & 8   & 11:53              \\\hline
  PVS-Studio        & 4   & 0   & 10:36              \\\hline
  Splint            & --- & --- & ---                \\\hline
  Li et at.         & 7   & 202 & 37:55              \\\hline
  Улучшенная версия & 7   & 4   & 42:49              \\\hline
  \end{tabular}

  TP (true positive) --- число корректно найденных ошибок.

  FP (false positive) --- число ложных срабатываний.

\end{frame}
\note {
  TODO
}

\begin{frame}{Сравнение: CMake}
  CMake 1.3.1: 350 тыс. строк кода, 31 мегабайт биткода.

  \begin{tabular}{|*{18}{c|}}\hline
  --                & TP  & FP  & Время работы (м:с) \\\hline
  Cppcheck          & 1   & 0   & 5:59               \\\hline
  PVS-Studio        & 1   & 0   & 5:18               \\\hline
  Splint            & --- & --- & ---                \\\hline
  Li et at.         & 5   & 79  & 9:12               \\\hline
  Улучшенная версия & 5   & 3   & 11:03              \\\hline
  \end{tabular}

  TP (true positive) --- число корректно найденных ошибок.

  FP (false positive) --- число ложных срабатываний.

\end{frame}
\note {
  TODO
}

\begin{frame}{Сравнение: синтетические тесты}

  Синтетические тесты: различные использования массива, 12 ошибочных.
  \begin{table}[h]
  \centering
  \begin{tabular}{|*{18}{c|}}\hline
  --                & TP  & FP & FN \\\hline
  Clang Analyzer    & 0   & 0  & 0  \\\hline
  CppCheck          & 3   & 0  & 9  \\\hline
  PVS-Studio        & 4   & 0  & 8  \\\hline
  Splint            & 9   & 3  & 3  \\\hline
  Li et at.         & 12  & 8  & 0  \\\hline
  Улучшенная версия & 12  & 0  & 0  \\\hline
  \end{tabular}
  \end{table}

  TP (true positive) --- число корректно найденных ошибок.

  FP (false positive) --- число ложных срабатываний.

  FN (false negative) --- число реальных ошибок, не найденных анализатором.
\end{frame}
\note { Также был разработан набор синтетических тестов, в которых
  содержатся различные ситуации обращения к массиву: как корректные,
  так и ошибочные. Clang Analyzer не нашёл ни одной ошибки. CppCheck и
  PVS-Studio проводят межпроцедурный анализ, однако смогли найти лишь
  3 и 4 ошибки соответственно. Из достоинств стоит отметить отсутствие
  ложных срабатываний. Split, в отличие от предыдущих двух, сообщает
  об ошибках всегда, когда не может доказать обратного. За счёт этого
  удалось найти 9 ошибок, однако так же было 3 ложных срабатывания,
  одно из них — из-за отсутствия межпроцедурного анализа. Также splint
  умеет работать только с C кодом, но не C++. Версия анализатора,
  реализованная в точном соответстии с упомянутой ранее статьёй,
  успешно нашла все 12 ошибок, однако также было 8 ложных срабатываний
  по причинам, описанным выше. Наконец, улучшенная версия также нашла
  все ошибки, но без единого false positive.  }

\begin{frame}{Заключение}
    \begin{itemize}
        \item За основу взят известный подход из статьи Li et al.
        \item Выявлены недостатки подхода.
        \item Сделаны улучшения, направленные на устранение выявленных
          недостатков.
        \item Получился анализатор, способный находить
          ошибки в больших программах, работающий лучше доступных
          анализаторов в некоторых случаях.
    \end{itemize}
\end{frame}
\note { Таким образом, в рамках данной работы за основу был взят
  известный легковесный подход, были выявлены различные недостатки и
  сделаны улучшения, направленные на устранение этих недостатков. В
  результате получился легковесный анализатор, способный находить
  ошибки в больших программах и работающий лучше доступных
  анализаторов в некоторых случаях.
}

\begin{frame}{Спасибо за внимание!}
    \begin{center}
        \Huge
        {\color{blue} Вопросы?}
    \end{center}
\end{frame}
\note{До свидания!}

% \appendix

% \begin{frame}[noframenumbering, t]{Обоснование метрик}
%     \only<1> {
%       kek
%     }
%     \only<2> {
%       хи-хи
%     }
% \end{frame}

\end{document}
