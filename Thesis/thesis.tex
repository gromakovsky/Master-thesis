\documentclass[specification,annotation,times]{itmo-student-thesis}

%% Опции пакета:
%% - specification - если есть, генерируется задание, иначе не генерируется
%% - annotation - если есть, генерируется аннотация, иначе не генерируется
%% - times - делает все шрифтом Times New Roman, требует пакета pscyr.

%% Делает запятую в формулах более интеллектуальной, например:
%% $1,5x$ будет читаться как полтора икса, а не один запятая пять иксов.
%% Однако если написать $1, 5x$, то все будет как прежде.
\usepackage{icomma}

%% Данные пакеты необязательны к использованию в бакалаврских/магистерских
%% Они нужны для иллюстративных целей
%% Начало
\usepackage{tikz}
\usetikzlibrary{arrows}
\usepackage{placeins}
\usepackage{subfig}
% \usepackage{filecontents}

%% Конец

%% Указываем файл с библиографией…
\addbibresource{thesis.bib}

%% …и папку с рисунками.
\graphicspath{ {../Images/} }

\newcommand{\inputTikZ}[1]{\input{../Images/#1.tikz}}

\begin{document}

%% Метаданные об авторе и работе. Нужны для генерации титульника,
%% аннотации и задания.
\studygroup{M4239}
\title{Разработка алгоритмов статического поиска выходов за пределы
    динамического массива в С/C++ программах}
\author{Громаковский Иван Евгеньевич}{Громаковский И.Е.}
\supervisor{Лукин Михаил Андреевич}{Лукин М.А.}
    {канд. техн. наук}{тьютор кафедры КТ Университета ИТМО}
\publishyear{2017}
%% Дата выдачи задания. Можно не указывать, тогда надо будет заполнить от руки.
\startdate{01}{сентября}{2015}
%% Срок сдачи студентом работы. Можно не указывать, тогда надо будет заполнить от руки.
\finishdate{31}{мая}{2017}
%% Дата защиты. Можно не указывать, тогда надо будет заполнить от руки.
%% \defencedate{15}{июня}{2015}

%% \addconsultant{Белашенков Н.Р.}{канд. физ.-мат. наук, без звания}
%% \addconsultant{Беззубик В.В.}{без степени, без звания}

%% Задание
%%% Техническое задание и исходные данные к работе
\technicalspec{Требуется разработать и реализовать алгоритм
  статического поиска выходов за пределы динамического массива в C/C++
  программах. Необходимо обрабатывать промышленные программы больших
  размеров (состоящие из сотен тысяч строк) за разумное время (порядка
  нескольких десятков минут). Анализатор должен находить как можно
  больше ошибок с как можно меньшим числом ложных срабатываний. На
  сегодняшний день известно множество свободно распространяемых
  анализаторов, решающих аналогичную или более общую проблему. В
  данной работе необходимо также произвести сравнение с другими
  анализаторами.}

%%% Содержание выпускной квалификационной работы (перечень подлежащих разработке вопросов)
\plannedcontents{\begin{enumerate}
    \item Обоснование актуальности задачи, описание предметной
      области, обзор существующих решений.
    \item Теоретическое исследование задачи поиска выходов за пределы
      динамического массива в C/C++. Разработка алгоритма.
    \item Описание практической реализации алгоритма, результаты
      экспериментов, сравнение с другими анализаторами.
\end{enumerate}}

%%% Исходные материалы и пособия
\plannedsources{\begin{enumerate}
    \item Li, Lian, Cristina Cifuentes, and Nathan Keynes. "Practical
      and effective symbolic analysis for buffer overflow detection."
      Proceedings of the eighteenth ACM SIGSOFT international
      symposium on Foundations of software engineering. ACM, 2010.
    \item Shahriar, Hossain, and Mohammad Zulkernine. "Classification
      of static analysis-based buffer overflow detectors." Secure
      Software Integration and Reliability Improvement Companion
      (SSIRI-C), 2010 Fourth International Conference on. IEEE, 2010.
    \item Le, Wei, and Mary Lou Soffa. "Marple: a demand-driven
      path-sensitive buffer overflow detector." Proceedings of the
      16th ACM SIGSOFT International Symposium on Foundations of
      software engineering. ACM, 2008.
    \item Ding, Sun, et al. "Detection of buffer overflow
      vulnerabilities in C/C++ with pattern based limited symbolic
      evaluation." Computer Software and Applications Conference
      Workshops (COMPSACW), 2012 IEEE 36th Annual. IEEE, 2012.
\end{enumerate}}

%%% Календарный план
\addstage{Ознакомление с основами статического анализа}{10.2015}
\addstage{Изучение существующих подходов к решению данной проблемы}{02.2016}
\addstage{Разработка и реализация прототипа анализатора}{06.2016}
\addstage{Исследование преимуществ и недостатков прототипа,
  разработка теоретических улучшений}{11.2016}
\addstage{Реализация улучшенной версии анализатора}{01.2017}
\addstage{Проведение экспериментов, сравнение с другими анализаторами}{02.2017}
\addstage{Написание пояснительной записки}{05.2017}

%% Аннотация
%%% Цель исследования
\researchaim{Разработка удобного стилевого файла \LaTeX
             для бакалавров и магистров кафедры компьютерных технологий.}

%%% Задачи, решаемые в ВКР
\researchtargets{\begin{enumerate}
    \item соответствие титульной страницы, задания и аннотации шаблонам, принятым в настоящее время на
    кафедре;
    \item соответствие содержательной части пояснительной записки требованиям ГОСТ~7.0.11-2011 <<Диссертация и
    автореферат диссертации>>;
    \item относительное удобство в использовании~--- указание данных об авторе и научном руководителе один раз
    и в одном месте, автоматический подсчет числа тех или иных источников.
\end{enumerate}}

%%% Использование современных пакетов компьютерных программ и технологий
\advancedtechnologyusage{Была использована система компьютерной верстки \LaTeX, а в
рамках нее следующие пакеты, в порядке появления в стилевом файле: babel, csquotes,
geometry, amsmath, amssymb, amsthm,
amsfonts, amsextra, graphicx, xcolor, colortbl, tabu, caption, floatrow, algorithm,
algorithmicx, algpseudocode, enumitem, setspace, biblatex (а именно biber), lastpage,
totcount, longtable, listings, chngcntr, titlesec, titletoc, ifpdf.}

%%% Краткая характеристика полученных результатов
\researchsummary{Получился, надо сказать, практически неплохой стилевик. В 2015 году
его уже использовали некоторые бакалавры и магистры. Надеюсь на продолжение.}

%%% Гранты, полученные при выполнении работы
\researchfunding{Автор разрабатывал этот стилевик исключительно за свой счет и на
добровольных началах. Однако значительная его часть была бы невозможна, если бы
автор не написал в свое время кандидатскую диссертацию в \LaTeX,
а также не отвечал за формирование кучи научно-технических отчетов по гранту,
известному как <<5-в-100>>, что происходило при государственной финансовой поддержке
ведущих университетов Российской Федерации (субсидия 074-U01).}

%%% Наличие публикаций и выступлений на конференциях по теме выпускной работы
\researchpublications{По теме этой работы я (к счастью!) ничего не публиковал.
\begin{refsection}
Однако покажу, как можно ссылаться на свои публикации из списка литературы:
\nocite{example-english, example-russian}
\printannobibliography
\end{refsection}
}


%% Эта команда генерирует титульный лист, задание и аннотацию.
\maketitle{Магистр}

%% Оглавление
\tableofcontents

%% Макрос для введения. Совместим со старым стилевиком.
\startprefacepage

%% В этом файле написано введение.
% -*-coding: utf-8-*-

Программное обеспечение всегда было и остаётся подвержено
ошибкам. Одними из наиболее уязвимых являются программы, написанные на
языках C и C++, поскольку они предоставляют прямой доступ к памяти и
не имеют встроенных механизмов для предотвращения некорректного
обращения к памяти. Пожалуй, наибольшую опасность представляют ошибки,
при которых происходит доступ к памяти (чтение или запись) за
пределами выделенного буфера. Такие ошибки могут приводить как к
остановке программы, так и к более серьёзным проблемам с точки зрения
безопасности. Например, злонамеренный пользователь может получить
доступ к приватной информации, исполнить произвольный код, а в худшем
случае получить права суперпользователя~\cite{onesmashing}.

Существует большое число подходов к решению данной проблемы, как
статических~\cite{wagner2000first, xie2003archer, ganapathy2003buffer,
le2008marple, li2010practical}, так и
динамических~\cite{cowan1998stackguard,
ruwase2004practical}. Динамические подходы обнаруживают ошибки во
время работы программы и имеют несколько серьёзных недостатков по
сравнению со статическими подходами, когда программа анализируется до
запуска. Во-первых, дополнительные проверки, выполняемые во время
работы программы, занимают какое-то время и тем самым ухудшают
производительность. Во-вторых, даже если ошибка будет обнаружена,
программа в лучшем случае просто остановится. В-третьих, зачастую
оказывается довольно сложно быстро внести изменения, направленные на
исправление ошибок, и донести их до конечного пользователя. Поэтому
гораздо предпочтительнее обнаружить ошибки до поставки программы
пользователю.

Основная цель данной работы состоит в разработке алгоритмов
статического анализа C/C++ программ на предмет выхода за пределы
динамического массива. При этом необходимо обрабатывать большие
программы, состоящие из сотен тысяч строк, за разумное время. В общем
случае задача поиска всех выходов за пределы массива без единого
ложного срабатывания неразрешима, поэтому анализатор должен находить
как можно больше реальных ошибок, при этом выдавая как можно меньше
ложных срабатываний.

За основу данной работы был взят подход, описанный в
статье~\cite{li2010practical}. Были выявлены преимущества и недостатки
предложенного подхода, предложены и реализованы способы решения
найденных недостатков. С помощью проделанных улучшений удалось
существенно уменьшить число ложных срабатываний, тем самым увеличив
точность анализа. Также было проведено сравнение с другими свободно
распространяемыми анализаторами, показано превосходство над
ними. Реализованный алгоритм способен обрабатывать программы,
состоящие из сотен тысяч строк, за время порядка десяти минут.

В первой главе приведён обзор предметной области и существующих
решений описанной проблемы. Во второй главе представлены результаты
исследования алгоритма из статьи~\cite{li2010practical}, описан
теоретический подход к устранению выявленных недостатков. В третьей
главе рассмотрены вопросы практической реализации предложенного
решения, приведены основные результаты работы и сравнение с другими
анализаторами.


%% Начало содержательной части.
%% В этих файликах находятся главы из содержательной части.

%-*-coding: utf-8-*-
\chapter{Обзор предметной области}

\section{Описание задачи}

TODO

\section{Обзор существующих решений}

TODO

\section{Обзор статьи~\cite{li2010practical}}

TODO

\chapterconclusion

В конце каждой главы желательно делать выводы. TODO

%-*-coding: utf-8-*-

\chapter{Разработанные улучшения}

В данной работе было проведено подробное исследование подхода,
представленного в статье~\cite{li2010practical}. В связи с отсутствием
реализации описанного алгоритма в открытом доступе он был реализован с
нуля в полном соответствии с приведённым в статье описанием. В ходе
исследования основное внимание было уделено недостаткам этого
подхода. Были выявлены случаи, в которых описанный алгоритм работает
некорректно. Ниже представлено описание таких случаев, а также
предложены способы решения этих проблем.

\section{Обработка циклов}

В ходе проведённого исследования были выявлены две проблемы базового
алгоритма, связанные с обработкой циклов. Эти проблемы приводят к
ложным срабатываниям. Ниже представлено более подробное описание
недостатков и предложены улучшения, направленные на их устранение.

\subsection{Монотонно изменяющиеся переменные}

\begin{figure}
    \includegraphics[]{for-trivial.png}
    \caption{Простой цикл: C++}
    \label{fig:for-trivial-cpp}
\end{figure}

\begin{figure}
    \inputTikZ{for-trivial}
    \caption{Простой цикл: SSA форма}
    \label{fig:for-trivial-ssa}
\end{figure}

На рисунке~\ref{fig:for-trivial-cpp} представлен фрагмент C++ кода с
простым циклом, в котором значение переменной $x$ меняется от нуля до
девяти включительно. На рисунке~\ref{fig:for-trivial-ssa} представлена
соответствующая ему SSA форма.

В данном случае алгоритм, описанный в статье~\cite{li2010practical},
будет работать следующим образом. Для проверки корректности записи в
массив будет посчитан диапазон возможных значений переменной $x$ в
месте записи в массив. Для этого сначала будет посчитан диапазон
значений $x$ в месте определения. Изначально в качестве диапазона
будет взят $[\bot, \top]$. Поскольку $x$ выражается как
$\phi$-инструкция, необходимо посчитать диапазон значений $x_1$ в
месте определения $x$ и объединить с $[0, 0]$ (диапазоном значений
второго аргумента). Диапазон значений $x_1$ вычисляется через диапазон
значений $x$ в месте определения $x_1$. Изначально для $x$ был взят
отрезок $[\bot, \top]$, однако в месте определения $x_1$ также будет
применён предикат $x < 10$. Таким образом, диапазон значений $x_1$
будет $[\bot, 10]$. Объединяя его с $[0, 0]$, получаем $[\bot, 10]$ и
для $x$. Это значит, что в данном примере анализатор посчитает, что
$x$ может быть отрицательным, а значит, запись в массив
некорректна. Однако нетрудно видеть, что в действительности это не
так, $x$ не может быть меньше нуля, а запись в массив всегда
корректна. Таким образом, алгоритм выдаёт ложное срабатывание на
данном примере.

Для решения проблемы предлагается использовать дополнительное правило
для вычисления define range. Предположим, что значение $x$ определено
как $\phi(a, b)$, при этом $b = f(x)$. Пусть $f$ удовлетворяет
условию, что последовательность $a, f(a), f(f(a)), \dots$ ---
монотонна (не умаляя общности, предположим, что последовательность
возрастает). Нетрудно видеть, что в таком случае множество возможных
значений $x$ содержится в $\{f^i(a) : i \in [0 .. \inf)\}$ и что все
элементы этого множества не меньше $a$.  Тогда для вычисления define
range $x$ применяется условие $x \geq a$. Примером такой функции $f$
является прибавление или вычитание значения с постоянным знаком.

За счёт использования описанного выше правила в приведённом примере
define range переменной $x$ будет $[0, 10]$, т. к. прибавление единицы
удовлетворяет сформулированному выше условию. Use range в инструкции
записи в массив будет $[0, 9]$. Таким образом, в данном случае удаётся
избежать ложного срабатывания.

\subsection{Обработка предиката «не равно»}

\begin{figure}
    \includegraphics[]{for-ne.png}
    \caption{Цикл с условием «не равно»: C++}
    \label{fig:for-ne-cpp}
\end{figure}

\begin{figure}
    \inputTikZ{for-ne}
    \caption{Цикл с условием «не равно»: SSA форма}
    \label{fig:for-ne-ssa}
\end{figure}

На рисунке~\ref{fig:for-ne-cpp} представлен фрагмент C++ кода с
циклом, аналогичным представленному на
рисунке~\ref{fig:for-trivial-cpp}, но в котором значение счётчика
ограничено с помощью условия «не равно». Такие циклы очень часто
встречаются в программах.  На рисунке~\ref{fig:for-ne-ssa}
представлена соответствующая ему SSA форма.

Подход из статьи~\cite{li2010practical} учитывает предикат $x \neq y$
при вычислении use range переменной $v$ только в том случае, если
равенство $x = y$ выполняется для граничного значения $v$. В таком
случае диапазон значений $v$ сокращается на одно значение. Из-за этого
алгоритм неспособен корректно обрабатывать циклы, в которых значение
счётчика ограничено таким предикатом. В примере на
рисунке~\ref{fig:for-ne-ssa} диапазон значений $x$ без учёта предиката $x \neq
10$ будет $[0, \top]$. Равенство $x = 10$ не соответствует
граничному значению, значит, предикат будет проигнорирован. В
результате алгоритм посчитает, что запись в массив может выйти за
границы, хотя легко видеть, что это не так.

Решение данной проблемы похоже на решение предыдущей проблемы и
заключается в введении дополнительного правила для вычисления use
range. Предположим, что значение $x$ определено как $\phi(a, b)$, при
этом $b = f(x)$. Пусть $f$ снова удовлетворяет условию, что
последовательность $a, f(a), f(f(a)), \dots$ --- монотонна (опять
считаем, что последовательность возрастает). Предположим, что условие
$x \neq y$ всегда выполняется в инструкции $P$, для которой
вычисляется use range переменной $x$. Пусть существует такое $c$, что
$y = f^c(a)$. Нетрудно видеть, что последовательность значений $x$
задаётся как $x_i = f^i(a)$. Из условия существования $c: y = f^c(a)$
и предиката $x \neq y$ следует, что всего может быть не более чем $c$
значений $x$. Из свойства функции $f$ следует, что $x < f^c(a) =
y$. Таким образом, при вычислении use range применяется предикат
$x < y$.

В примере на рисунке~\ref{fig:for-ne-ssa} $f(x) = x + 1$, $y = 10$,
$a = 0$, $c = 10$. Таким образом, анализатор способен вывести, что
$x < 10$, за счёт чего обращение к массиву будет обработано корректно.

\FloatBarrier

\section{Учёт предикатов}

\begin{figure}
    \includegraphics[]{predicates-improvement.png}
    \caption{Цикл с дополнительным условием внутри: C++}
    \label{fig:predicates-improvement-cpp}
\end{figure}

\begin{figure}
    \inputTikZ{predicates-improvement}
    \caption{Цикл с дополнительным условием внутри: SSA форма}
    \label{fig:predicates-improvement-ssa}
\end{figure}

В цикле, представленном на
рисунке~\ref{fig:predicates-improvement-cpp}, значение переменной $x$,
использующейся как индекс при записи в массив, дополнительно
ограничено числом $7$. Таким образом, значение счётчика цикла
находится в диапазоне $[0, 9]$, однако значение индекса не превышает
$7$. Программа в SSA форме представлена на
рисунке~\ref{fig:predicates-improvement-ssa}.

Проблема подхода~\cite{li2010practical} состоит в правиле,
определяющем, какие предикаты должны учитываться при вычислении use
range переменной $v$ в инструкции $P$. Во-первых, условный переход,
использующий предикат, должен доминировать инструкцию $P$, то есть все
пути из входа в функцию в $P$ должны проходить через
предикат. Во-вторых, $P$ должна быть достижима только из одного
потомка условного перехода. Однако, как нетрудно видеть из
рисунка~\ref{fig:predicates-improvement-ssa}, в данном случае
инструкция записи в массив достижима из обоих потомков условного
перехода. Это приводит к тому, что условие $x < 7$ не применяется для
вычисления use range $x$ в месте записи в массив, из-за чего
анализатор выдаёт несуществующую ошибку.

Описанная проблема может быть решена модификацией второго правила,
определяющего, должен ли учитываться предикат при вычислении use
range. Модифицированное правило ослабляет требование, что инструкция
$P$ должна быть достижима только из одного потомка условного перехода,
использующего предикат. Ослабленное требование заключается в том, что
должен быть ровно один потомок $S$ условного перехода $C$, такой что
$P$ достижима из $C$, игнорируя все рёбра из $C$, кроме
$C \rightarrow S$. В таком случае условие, соответствующее переходу в
$S$, считается выполненым.

Покажем, что предложенная модификация сохраняет
корректность. Рассмотрим произвольный путь в инструкцию $P$. По
первому правилу $P$ доминируется $C$, а значит, путь обязан пройти
через $C$. Рассмотрим последнее ребро на этом пути, исходящее из
$C$. Пусть это ребро $C \rightarrow T$. Если $T = S$, значит, условие,
соответствующее переходу в $S$, выполнено, что и требовалось
доказать. Если же $T \neq S$, то путь в $P$ проходит по другому ребру
из $C$, а значит, предположение о том, что ребро $C \rightarrow T$
является последним на пути, неверно.

При использовании модифицированного правила предикат $x < 7$ будет
учтён в месте записи в массив, т. к. из вершины $x_1 = x + 1$ нет
пути, не проходящего по ребру, соответствующему условию $x < 7$. Таким
образом, предложенная модификация позволяет избавиться от ложного
срабатывания в приведённом примере, при этом сохраняя общую
корректность анализа.

\FloatBarrier

\section{Межпроцедурный анализ}

В статье~\cite{li2010practical} описывается лишь алгоритм без
межпроцедурного анализа. Каждая функция анализируется независимо,
информация о возможных значениях аргументов игнорируется. Каждый
аргумент рассматривается точно так же, как и любое значение,
приходящее извне. На практике программы обычно состоят из большого
числа маленьких функций, и анализа одной изолированной функции
недостаточно, чтобы точно определить, всегда ли обращение к массиву
корректно.

В данной работе анализ был расширен до межпроцедурного. В целом
известно два общих подхода к межпроцедурному анализу: в одном подходе
при анализе вызывающей стороны определяются возможные значения
аргументов и используются для анализа вызываемой функции, в другом
подходе при анализе вызываемой функции определяются условия, которым
должны соответствовать аргументы, и проверяются при непосредственном
вызове. В данной работе используется второй вариант. Основная идея
описана в статье~\cite{xie2003archer}. Функции анализируются в порядке
топологической сортировки: от вызываемой к вызывающей. Если в графе
вызовов есть цикл (рекурсия), то топологическая сортировка
неопределена, и в таком случае цикл разрывается в случайных
местах. Также вводится понятие триггера, который задаёт условие,
влияющее на факт выхода за пределы массива. При анализе вызываемой
функции формируется множество триггеров. При анализе вызова функции
проверяется выполнимость каждого триггера.

\subsection{Триггеры}

Триггер задаётся упорядоченной парой символьных выражений $e_1$ и
$e_2$. Оба выражения должны состоять только из констант и аргументов
функции (включая их афинные преобразования). Смысл триггера в том, что
если $e_1 \prec e_2$, то это приведёт к выходу за пределы
массива. Также триггер должен хранить внутри себя инструкцию, в
которой может произойти выход за пределы массива, чтобы при его
срабатывании можно было понять, в каком месте случится ошибка.

\begin{figure}
    \includegraphics[]{archer-linux.png}
    \caption{Межпроцедурный анализ}
    \label{fig:archer-linux}
\end{figure}

Для иллюстрации рассмотрим упрощённый фрагмент кода из ядра Линукс
версии 2.5.53, представленный на рисунке~\ref{fig:archer-linux}. В
данном примере функция \texttt{get\_slot\_by\_minor} вызывает функцию
\texttt{get\_drv\_by\_nr}. При анализе \texttt{get\_drv\_by\_nr} будет произведено
построение триггеров для этой функции. Затем эти триггеры будут
проверены при анализе \texttt{get\_slot\_by\_minor} (она будет анализирована
после \texttt{get\_drv\_by\_nr}, поскольку функции анализируются в порядке
топологической сортировки). Нетрудно видеть, что в данном случае выход
за пределы массива в функции \texttt{get\_drv\_by\_nr} может произойти тогда
и только тогда, когда $ISDN\_MAX\_DRIVERS \leq di \vee di \leq
-1$. Таким образом, для функции должно быть построено два триггера,
соответствующих этим неравенствам. В функции \texttt{get\_slot\_by\_minor}
происходит вызов \texttt{get\_drv\_by\_nr}, и в месте вызова можно расчитать
диапазон возможных значений аргумента, передаваемого в
\texttt{get\_slot\_by\_minor}. За счёт этого можно вывести, что один из
триггеров выполняется в месте вызова, а значит, в программе есть
ошибка.

Использование триггеров позволяет учесть информацию о значениях
аргументов функций, а также о завимостях между этими аргументами. Ниже
представлено более подробное описание использования триггеров.

\subsection{Построение триггеров}

Построение триггеров происходит при определении корректности доступа к
массиву в тех случаях, когда одно или несколько рассматриваемых
символьных значений зависят от аргументов функции. Как уже было
сказано в первой главе, выход за пределы массива размера $n$ при
обращении по индексу $index$ (инструкция $P$) фиксируется, если
$S_{n, P}^{max} \prec S_{index, P}^{max}$ или
$S_{index, P}^{min} \prec -1$. Таким образом, для проверки
используются диапазоны $n$ и $index$. Если $S_{n, P}^{max}$ или
$S_{index, P}^{max}$ зависит от аргумента функции, то их сравнение в
общем случае невозможно. В таком случае для функции добавляется
триггер $S_{n, P}^{max} \prec S_{index, P}^{max}$, хранящий также
инструкцию $P$. Выполнение этого триггера означает выход за пределы
массива в результате исполнения инструкции $P$. Аналогично
обрабатывается условие $S_{index, P}^{min} \prec -1$. В результате
обработки функции все сгенерированные триггеры сохраняются в
глобальном контексте анализатора для дальнейшей проверки в месте
вызова. Если при проверке был сгенерирован хотя бы один триггер, то
выход за пределы массива не фиксируется, а проверка откладывается для
вызывающей стороны.

Вернёмся к примеру, представленному на
рисунке~\ref{fig:archer-linux}. При анализе обращения к массиву
$drivers$ внутри функции \texttt{get\_drv\_by\_nr} (назовём эту инструкцию
$P$) необходимо проверить два условия: $32 \prec S_{di, P}^{max}$ и
$S_{di, P}^{min} \prec -1$. Поскольку $di$ является аргументом
функции, оба условия приведут к созданию триггера для рассматриваемой
функции, ассоциированных с инструкцией $P$. Триггеры будут сохранены и
проверены во время анализа функции \texttt{get\_slot\_by\_minor}.

Описанное правило применяется в тех случаях, когда рассматриваемые
символьные выражения зависят только от констант и аргументов
функции. Точно так же обрабатываются случаи, когда символьное
выражение задаётся функцией от нескольких аргументов. Например, если
бы функция принимала также аргумент $si$, а обращение происходило по
индексу $2 * si - di$, то такое символьное выражение тоже привело бы к
созданию триггера. За счёт этого учитываются зависимости между
аргументами, которые нередко имеют место в сложных функциях.

\subsection{Проверка триггеров}

При анализе инструкции вызова функции происходит проверка выполнимости
триггеров, сохранённых для этой функции. Напомним, что триггер состоит
из двух символьных выражений, в которых могут присутствовать аргументы
вызываемой функции. Каждому аргументу вызываемой функции соответствует
какое-то значение из вызывающей функции. Для каждого такого значения
рассчитывается диапазон значений в месте вызова функции (use
range). Таким образом, в общем случае символьным выражениям $e_1$ и
$e_2$, сравниваемым для проверки триггера, соответствуют символьные
отрезки $r_1$ и $r_2$. После вычисления $r_1$ и $r_2$ проверка
триггера происходит аналогично проверке выходу за пределы массива в
простом случае. Если $r_1^{max} \prec r_2^{min}$, то триггер всегда
выполняется, и в таком случае фиксируется ошибка. Если
$r_2^{max} \prec r_1^{min} + 1$, то ошибки точно нет. В противном
случае ошибка потенциально может произойти, но не точно. По умолчанию
в спорных ситуациях ошибка не фиксируется для избежания слишком
большого числа ложных срабатываний.

В примере на рисунке~\ref{fig:archer-linux} при анализе вызова функции
\texttt{get\_drv\_by\_nr} из функции \texttt{get\_slot\_by\_minor} проверяются два
триггера, построенные при анализе \texttt{get\_drv\_by\_nr}:
$32 \prec S_{di, call}^{max}$ и $S_{di, call}^{min} \prec -1$, где
$di$ является аргументом \texttt{get\_drv\_by\_nr}, а $call$ --- инструкция
вызова функции. В данном случае диапазон $S_{di, call}$ будет вычислен
как $[0, 64]$. В результате будет выполнено условие первого триггера,
что приведёт к сообщению об ошибке. За счёт хранения инструкции, в
которой происходит обращение к массиву, вместе с триггером анализатор
сможет сообщить и место вызова функции, и место внутри функции, в
котором случается ошибка.

\FloatBarrier

\chapterconclusion

В главе 2 представлены результаты исследования алгоритма,
описанного в статье~\cite{li2010practical}. В ходе исследования были
выявлены различные недостатки алгоритма, которым было уделено внимание
в данной работе. Показаны примеры кода, когда алгоритм работает
некорректно, предложены модификации исходного алгоритма, направленные
на устранение выявленных проблем. Доказана корректность предложенных
модификаций.

Одним из наиболее значимых улучшений является использование
межпроцедурного анализа, что крайне важно для анализа промышленных
программ. Предложенный подход базируется на идеях, использованных в
статье~\cite{xie2003archer}, однако адаптирован для алгоритма из
статьи~\cite{li2010practical}.

%-*-coding: utf-8-*-
\chapter{Практическая реализация и результаты}

\section{Реализация}

Предложенный подход был реализован на языке C++.  Как и в
статье~\cite{li2010practical}, было принято решение анализировать не
исходный код на C/C++, а промежуточное представление
LLVM~\cite{lattner2004llvm}. У такого подхода есть несколько весомых
преимуществ~\cite{merz2012llbmc}. Во-первых, LLVM-IR состоит из более
простых языковых конструкций, нежели C или тем более C++, что упрощает
анализ и позволяет охватить большее число возможностей языка с
меньшими усилиями. Во-вторых, код LLVM-IR представляет из себя
результат работы компилятора и является очень близким к тому, что
будет реально выполняться. Это позволяет найти ошибки, появившиеся в
результате трансформаций, выполняемых компилятором. Также
инфраструктура LLVM содержит огромное число встроенных оптимизаций,
средств для анализа и т. п., что можно использовать при
реализации. Ещё одним преимуществом анализа LLVM-IR является то, что
за счёт этого автоматически поддерживается любой язык, для которого
есть компилятор в LLVM-IR, не только C и C++ (являющиеся примерами
таких языков). Также стоит отметить, что программа в LLVM-IR всегда
представлена в SSA форме, на которой базируется описанный ранее
алгоритм. Альтернативное решение состоит в построении SSA формы для
исходной программы на C/C++ и анализе её, однако было показано, что
анализ LLVM-IR имеет существенные преимущества. Язык C++ был выбран
для реализации анализатора преимущественно по двум
причинам. Во-первых, библиотека LLVM написана на C++ и может быть
использована напрямую. Во-вторых, за счёт использования C++ проще
достичь высокой производительности.

Изначально подход из статьи~\cite{li2010practical} был реализован в
полном соответствии с описанием. После этого были найдены недостатки и
разработаны способы их устранения, описанные в предыдущей главе.

\subsection{LLVM}

В LLVM каждое значение ($Value$) идентифицируется указателем
($Value *$), поэтому диапазоны переменных ассоциировались с
указателями на значения. Для идентификации инструкций в рамках use
range использовались указатели на базовые блоки, поскольку use range
переменной одинаков во всех инструкциях из одного базового блока. Для
выявления инструкций, выделяющих блоки памяти, использовалась функция
$isAllocationFn$ из LLVM. Размер массива вычисляется как число байт,
выделяемых такой функций, поделённое на размер типа данных в массиве.

TODO

\subsection{Gated Single Assignment}

Одной из наиболее важных деталей алгоритма является использование
Gated Single Assignment формы
(GSA)~\cite{ottenstein1990program}. Отличие этой формы от SSA
заключается в том, что аргументы $\phi$-инструкций также содержат
условия, которые гарантировано выполняются, если данное значение
возвращается $\phi$-инструкцией. Это позволяет существенно повысить
точность анализа, отбрасывая невозможные значения переменных.

В реализации анализатора использовался алгоритм для конвертации SSA
формы в GSA, предложенный в статье~\cite{tu1995efficient}. Данный
алгоритм является одним из наиболее простых в реализации. Он не
требует, чтобы исходная программа была в SSA, однако существенно
упрощается, если это выполнено. В основе алгоритма лежит понятие
«выражение пути» (path expression), которое является регулярным
выражением над алфавитом, состоящем из рёбер в графе потока
управления. Пути, приходящие в $\phi$-инструкцию, представляются
выражениями пути. В конце работы алгоритма выражения пути превращаются
в предикаты GSA-формы. Несмотря на свою простоту, алгоритм также
является более эффективным по сравнению со своими аналогами. В
результате работы алгоритма для каждого операнда каждой
$\phi$-инструкции известен определённый набор условий, который должны
выполняться, чтобы данное значение было результатом $\phi$-инструкции.

TODO: expand this subsection, add more subsections (at least one more
about something general).

\section{Экспериментальные результаты}

TODO

\section{Сравнение}

Также было проведено сравнение анализатора с другими доступными
анализаторами C/C++ кода. Использовались следующие анализаторы: Clang
Analyzer, CppCheck, PVS-Studio, Splint, а также реализация
анализатора, описанного в статье~\cite{li2010practical}, и реализация
предложенного подхода. Для сравнения использовался набор синтетических
тестов, представленных в листинге~\ref{lst:comparison}.

\begin{table}[!h]
\caption{Результаты сравнения анализаторов}\label{tab:comparison}
\centering
% \begin{tabu}{|*{18}{X[c]|}}\hline
%   --                 & TP  & FP & FN \\\hline
%   Clang Analyzer     & 0   & 0  & 0  \\\hline
%   CppCheck           & 3   & 0  & 9  \\\hline
%   PVS-Studio         & 4   & 0  & 8  \\\hline
%   Splint             & 9   & 3  & 3  \\\hline
%   Li et at.          & 12  & 8  & 0  \\\hline
%   Предложенный метод & 12  & 0  & 0  \\\hline
% \end{tabu}
  \begin{tabular}{|*{18}{c|}}\hline
  --                 & TP  & FP & FN \\\hline
  Clang Analyzer     & 0   & 0  & 0  \\\hline
  CppCheck           & 3   & 0  & 9  \\\hline
  PVS-Studio         & 4   & 0  & 8  \\\hline
  Splint             & 9   & 3  & 3  \\\hline
  Li et at.          & 12  & 8  & 0  \\\hline
  Предложенный метод & 12  & 0  & 0  \\\hline
  \end{tabular}
\end{table}

В таблице~\ref{tab:comparison} представлены результаты сравнения.

\chapterconclusion

В данной главе была вкратце описана реализация предложенного
подхода. Приведены экспериментальные результаты запуска анализатора на
крупных проектах. Показано, что анализатор способен находить в них
ошибки, работая за приемлимое время. Также было проведено сравнение с
доступными статическими анализаторами C/C++ кода на наборе тестов,
представленных в листинге~\ref{lst:comparison}. Продемонстрировано
превосходство над представленными анализаторами на данном наборе
тестов.


%% Макрос для заключения. Совместим со старым стилевиком.
\startconclusionpage

%-*-coding: utf-8-*-

Как было показано в данной работе, задача поиска выходов за пределы
динамического массива в C/C++ является крайне актуальной и
представляет практический интерес. Были рассмотрены различные подходы
к решению данной задачи, выделены основные характеристики различных
подходов. За основу был взят подход, описанный в
статье~\cite{li2010practical}.

В ходе работы было проведено детальное исследование
подхода~\cite{li2010practical}, выявлены и продемонстрированы
различные недостатки. Были предложены способы устранения выявленных
недостатков, показана корректность этих способов.

Алгоритм был реализован на языке C++. В результате экспериментов было
показано, что анализатор способен находить ошибки в крупных проектах и
работает за разумное время. Несмотря на наличие ложных срабатываний,
такой анализатор является полезным инструментом разработчика, особенно
операционных систем и сетевых программ. Также было проведено сравнение
с доступными анализаторами. Показано, что реализованный подход
работает лучше промышленных анализаторов, а также лучше базового
подхода, предложенного в \cite{li2010practical}.

Разработанное средство статического анализа было внедрено в проекте
«IoT Devices», реализуемом ООО~«Интеллектуальные системы».

\FloatBarrier


%% Обратите внимание на heading. Без него тоже работает, но название будет другим.
\printmainbibliography

\end{document}
