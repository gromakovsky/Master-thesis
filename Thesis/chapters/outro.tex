%-*-coding: utf-8-*-

Как было показано в данной работе, задача построения семейств
оптимальных маршрутов кораблей представляет достаточный практический
интерес. Были рассмотрены подходы к решению аналогичных задач,
выявлены их недостатки. После этого было формализовано описание
семейства оптимальных маршрутов. На основе формальной постановки
задачи были предложены различные эвристики для её решения. В
результате был разработан алгоритм для поиска семейств маршрутов и
реализован в виде программного модуля для решения данной задачи в
режиме реального времени.

Стоит заметить, что предложенный алгоритм является преимущественно
эвристическим, большинство принятых решений не имеют строгих
обоснований и основаны на различных исследованиях. Для оценки
проделанной работы разработанный программный модуль был отправлен на
экспертизу в ЗАО «ОСК-Транзас». Разработанные эвристики признаны
эффективными и разумными. Отмечено, что находимые маршруты
соответствуют заявленным требованиям. Эксперты рекомендуют
использовать алгоритм и критерии, заложенные в реализованный
программный модуль, для решения задач поддержки принятия решения в
навигационно-тактических системах морского назначения.

Также стоит отметить возможные направления дальнейших работ для
улучшения имеющегося решения:
\begin{itemize}
    \item В алгоритме активно применяется алгоритм Дейкстры для различных
      целей (поиск маршрута, вычисление потенциалов, вычисление метрик).
      Возможно, имеет смысл рассмотреть вопрос использования
      алгоритма~$A^*$ с целью потенциального улучшения производительности.
    \item Более сложной является задача поиска маршрутов при заданных
      запретных зонах в зависимости от времени. В этом случае
      известно, что в некоторые временные диапазоны в определённых
      местах будет нельзя плавать. Требуется находить маршруты,
      учитывая подобные ограничения.
\end{itemize}

Разработанный программный модуль был внедрён в имеющееся программное
обеспечение ЗАО~«Кронштадт Технологии».

\FloatBarrier
