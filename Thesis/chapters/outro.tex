%-*-coding: utf-8-*-

Как было показано в данной работе, задача поиска выходов за пределы
динамического массива в C/C++ является крайне актуальной и
представляет практический интерес. Были рассмотрены различные подходы
к решению данной задачи, выделены основные характеристики различных
подходов. За основу был взят подход, описанный в
статье~\cite{li2010practical}.

В ходе работы было проведено детальное исследование
подхода~\cite{li2010practical}, выявлены и продемонстрированы
различные недостатки. Были предложены способы устранения выявленных
недостатков, показана корректность этих способов.

Алгоритм был реализован на языке C++. В результате экспериментов было
показано, что анализатор способен находить ошибки в крупных проектах и
работает за разумное время. Несмотря на наличие ложных срабатываний,
такой анализатор является полезным инструментом разработчика, особенно
операционных систем и сетевых программ. Также было проведено сравнение
с доступными анализаторами. Показано, что реализованный подход
работает лучше промышленных анализаторов, а также лучше базового
подхода, предложенного в \cite{li2010practical}.

Разработанное средство статического анализа было внедрено в проекте
«IoT Devices», реализуемом ООО~«Интеллектуальные системы».

\FloatBarrier
