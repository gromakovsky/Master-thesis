\studygroup{M4239}
\title{Разработка алгоритмов статического поиска выходов за пределы
    динамического массива в С/C++ программах}
\author{Громаковский Иван Евгеньевич}{Громаковский И.Е.}
\supervisor{Лукин Михаил Андреевич}{Лукин М.А.}
    {канд. техн. наук}{тьютор кафедры КТ Университета ИТМО}
\publishyear{2017}
%% Дата выдачи задания. Можно не указывать, тогда надо будет заполнить от руки.
\startdate{01}{сентября}{2015}
%% Срок сдачи студентом работы. Можно не указывать, тогда надо будет заполнить от руки.
\finishdate{31}{мая}{2017}
%% Дата защиты. Можно не указывать, тогда надо будет заполнить от руки.
%% \defencedate{15}{июня}{2015}

%% \addconsultant{Белашенков Н.Р.}{канд. физ.-мат. наук, без звания}
%% \addconsultant{Беззубик В.В.}{без степени, без звания}

%% Задание
%%% Техническое задание и исходные данные к работе
\technicalspec{Требуется разработать и реализовать алгоритм
  статического поиска выходов за пределы динамического массива в C/C++
  программах. Необходимо обрабатывать промышленные программы больших
  размеров (состоящие из сотен тысяч строк) за разумное время (порядка
  нескольких десятков минут). Анализатор должен находить как можно
  больше ошибок с как можно меньшим числом ложных срабатываний. На
  сегодняшний день известно множество свободно распространяемых
  анализаторов, решающих аналогичную или более общую проблему. В
  данной работе необходимо также произвести сравнение с другими
  анализаторами.}

%%% Содержание выпускной квалификационной работы (перечень подлежащих разработке вопросов)
\plannedcontents{\begin{enumerate}
    \item Обоснование актуальности задачи, описание предметной
      области, обзор существующих решений.
    \item Теоретическое исследование задачи поиска выходов за пределы
      динамического массива в C/C++. Разработка алгоритма.
    \item Описание практической реализации алгоритма, результаты
      экспериментов, сравнение с другими анализаторами.
\end{enumerate}}

%%% Исходные материалы и пособия
\plannedsources{\begin{enumerate}
    \item Li, Lian, Cristina Cifuentes, and Nathan Keynes. "Practical
      and effective symbolic analysis for buffer overflow detection."
      Proceedings of the eighteenth ACM SIGSOFT international
      symposium on Foundations of software engineering. ACM, 2010.
    \item Shahriar, Hossain, and Mohammad Zulkernine. "Classification
      of static analysis-based buffer overflow detectors." Secure
      Software Integration and Reliability Improvement Companion
      (SSIRI-C), 2010 Fourth International Conference on. IEEE, 2010.
    \item Le, Wei, and Mary Lou Soffa. "Marple: a demand-driven
      path-sensitive buffer overflow detector." Proceedings of the
      16th ACM SIGSOFT International Symposium on Foundations of
      software engineering. ACM, 2008.
    \item Ding, Sun, et al. "Detection of buffer overflow
      vulnerabilities in C/C++ with pattern based limited symbolic
      evaluation." Computer Software and Applications Conference
      Workshops (COMPSACW), 2012 IEEE 36th Annual. IEEE, 2012.
\end{enumerate}}

%%% Календарный план
\addstage{Ознакомление с основами статического анализа}{10.2015}
\addstage{Изучение существующих подходов к решению данной проблемы}{02.2016}
\addstage{Разработка и реализация прототипа анализатора}{06.2016}
\addstage{Исследование преимуществ и недостатков прототипа,
  разработка теоретических улучшений}{11.2016}
\addstage{Реализация улучшенной версии анализатора}{01.2017}
\addstage{Проведение экспериментов, сравнение с другими анализаторами}{02.2017}
\addstage{Написание пояснительной записки}{05.2017}

%% Аннотация
%%% Цель исследования
\researchaim{Разработка удобного стилевого файла \LaTeX
             для бакалавров и магистров кафедры компьютерных технологий.}

%%% Задачи, решаемые в ВКР
\researchtargets{\begin{enumerate}
    \item соответствие титульной страницы, задания и аннотации шаблонам, принятым в настоящее время на
    кафедре;
    \item соответствие содержательной части пояснительной записки требованиям ГОСТ~7.0.11-2011 <<Диссертация и
    автореферат диссертации>>;
    \item относительное удобство в использовании~--- указание данных об авторе и научном руководителе один раз
    и в одном месте, автоматический подсчет числа тех или иных источников.
\end{enumerate}}

%%% Использование современных пакетов компьютерных программ и технологий
\advancedtechnologyusage{Была использована система компьютерной верстки \LaTeX, а в
рамках нее следующие пакеты, в порядке появления в стилевом файле: babel, csquotes,
geometry, amsmath, amssymb, amsthm,
amsfonts, amsextra, graphicx, xcolor, colortbl, tabu, caption, floatrow, algorithm,
algorithmicx, algpseudocode, enumitem, setspace, biblatex (а именно biber), lastpage,
totcount, longtable, listings, chngcntr, titlesec, titletoc, ifpdf.}

%%% Краткая характеристика полученных результатов
\researchsummary{Получился, надо сказать, практически неплохой стилевик. В 2015 году
его уже использовали некоторые бакалавры и магистры. Надеюсь на продолжение.}

%%% Гранты, полученные при выполнении работы
\researchfunding{Автор разрабатывал этот стилевик исключительно за свой счет и на
добровольных началах. Однако значительная его часть была бы невозможна, если бы
автор не написал в свое время кандидатскую диссертацию в \LaTeX,
а также не отвечал за формирование кучи научно-технических отчетов по гранту,
известному как <<5-в-100>>, что происходило при государственной финансовой поддержке
ведущих университетов Российской Федерации (субсидия 074-U01).}

%%% Наличие публикаций и выступлений на конференциях по теме выпускной работы
\researchpublications{По теме этой работы я (к счастью!) ничего не публиковал.
\begin{refsection}
Однако покажу, как можно ссылаться на свои публикации из списка литературы:
\nocite{example-english, example-russian}
\printannobibliography
\end{refsection}
}
