%-*-coding: utf-8-*-
\chapter{Обзор предметной области}

\section{Описание проблемы}

TODO

\section{Обзор существующих решений}

Известно большое число подходов к решению проблемы, описанной в
предыдущем разделе. Глобально их можно разделить на динамические и
статические.

Динамические подходы~\cite{cowan1998stackguard, ruwase2004practical,
  hastings1991purify} состоят в добавлении в программу дополнительных
проверок, предотвращающих обращение к памяти за пределами выделенного
буфера. Основное преимущество этого подхода в том, что он, как
правило, предотвращает большее число ошибок, поскольку значения всех
переменных известны в момент обращения к памяти. Однако такие подходы
существенно замедляют работу программы, что зачастую оказывается
неприемлимо в случаях, когда производительность стоит на первом
месте. Другим недостатком динамического подхода является тот факт, что
при наличии ошибки она будет обнаружена уже во время программы, что,
скорее всего, приведёт к остановке. Также динамические подходы
способны находить лишь ошибки, которые воспроизводятся на реально
выполняемых участках кода. На практике же нередко происходит так, что
какой-то фрагмент кода может не выполняться на протяжении очень
длительного времени, однако именно в нём может быть ошибка, которая
очень долго будет оставаться незамеченной.

Статический анализ, в отличие от динамического, производится без
запуска программы. Это позволяет исследовать все места в коде, даже
редко выполняемые, а также позволяет найти ошибки до реального
использования программы, пока они не проявили себя. Помимо этого
статический анализ никак не меняет исполняемый код, а значит, не
замедляет программу. В общем случае задача поиска всех выходов за
пределы массива в C/C++ без единого ложного срабатывания неразрешима
(это напрямую следует из проблемы
останова~\cite{turing1937computable}), поэтому одним из главных
недостатков статического анализа является необходимость ручного
рассмотрения результатов работы анализатора с целью выяснения, какие
из найденных ошибок действительно являются таковыми.

Таким образом, динамические и статические подходы имеют свои
преимущества и недостатки. В разных случаях имеет смысл применять
разные подходы (в том числе комбинацию подходов). Оба варианта
являются осмысленными и актуальными. В данной работе был сделан выбор
в пользу статического анализа.

TODO: написать про всякие конкретные статические подходы.

\section{Обзор статьи~\cite{li2010practical}}

На основании проведённого изучения литературы было принято решение
использовать алгоритм, описанный в статье \cite{li2010practical} в
качестве базового. Во-первых, этот подход является легковесным, что
позволяет достаточно быстро обрабатывать большие программы. Во-вторых,
результаты, представленные в статье, говорят о том, что анализатор
способен находить много ошибок в крупных открытых проектах. В-третьих,
несмотря на довольно хорошие результаты, авторы также отмечают наличие
ложных срабатываний и спорных ситуаций, в которых алгоритм не может
принять однозначное решение. А значит, у этого подхода есть
пространство для улучшений.

TODO: описать основную суть подхода из статьи.

\chapterconclusion

В данной главе была подробно описана проблема выхода за пределы
динамического массива в C/C++ программах, показана её
актуальность. Был приведён обзор существующих решений этой проблемы,
описаны ключевые характеристики различных статических подходов. Был
обоснован выбор алгоритма из статьи~\cite{li2010practical} в качестве
базового подхода для данной работы, приведён более подробный обзор
этого алгоритма.
